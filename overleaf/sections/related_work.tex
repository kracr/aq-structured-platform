\section{Related Work}
\label{sec:rel-work}

%In this section, we aim to highlight the works that have been done in the space of ontologies, integration practices and applications for air quality. The semantic web community in particular has not focused a lot on pollution or air quality or an integration of air quality with other kinds of data.

There are a few existing ontologies related to air pollution and air quality. Adeleke et al.,~\cite{adeleke} built an ontology to model pollution as a part of a web of IoT sensors. Their ontology is used in an indoor air pollution monitoring and control application. Mihaela et al.,~\cite{AIRPOLLUTIONONTO},  developed an ontology to model air pollution. They used human experts and heuristic rules generated through machine learning techniques as knowledge sources. They use the ontology to monitor and control air pollution in urban regions. Riga et al.,~\cite{marina_riga_2018_1196023} use ontologies to make efficient environmental decision support systems. Ontologies are used to model the status of the environment and its impact on citizens' health and actions. Mertal et al.,~\cite{metral} use CityGML and aim to connect 3D city models with air pollution ontologies for sustainable urban development. 

Some of the applications related to air pollution have focused on categorizing AQI value into good, poor and severe to generate recommendations on actions to be performed. Examples of such applications are CPCBcrr\footnote{CPCBcrr \url{https://cpcbccr.com/}} by CPCB, AQICN\footnote{\url{https://aqicn.org/}} and SAFAR\footnote{\url{http://safar.tropmet.res.in/}}. The forecasting system developed by SAFAR predicts air quality and generates recommendations a few days ahead. Urban Emissions~\footnote{\url{https://urbanemissions.info/}} analyses the contribution of components such as traffic, dust and wind on the AQI at a particular location. They use a chemical transport model system known as WRF-CAMx to integrate the data. All the applications discussed here do not generate spatiotemporally aware recommendations and do not consider a locality's social makeup. Instead, they rely on centrally located sensors that may not be able to capture the complete dynamics of air pollution at a location.

Air quality research generally involves analyzing pollution sources, government policies, and their adverse effect on public health. For example, Agarwala M. and Chandel A., explain the role of crop residue burning on Delhi's air quality~\cite{AGARWALA-2020-ERL} and Huang et al., studied the emission of pollutants from power generation plants~\cite{HUANG-2017-RCR}. We focus on social factors operating at the individual and institutional levels that affect the public's perception and awareness of pollution. We use an ontology based approach to integrate the air quality data with the ethnographic data of locations. %Air quality is closely related to factors such as wind, weather, geospatial formations, buildings and social settings. 

%The research literature around ontologies for air pollution/quality has largely been as a module in a larger ontology or as a simple ontology to show the dependence with pollutants. 
 %A set of rules dictates the calculation of indoor air quality index and thermal comfort index, the calculation can be performed by reasoning on the data modelled by their ontology. Some concepts like PM10Unhealthy in their ontology are dictated by axioms which perform logical operations on data properties. The values thus stored can then be fetched using SPARQL queries. 

%They create a 3D air quality model which reconstruct the environment, its properties and physical laws so that simulations can be make use of it. In the same pursuit, the authors use OUPP, cityGML (a 3D city model) and street canyon model - having numerous inputs such as pollutant source, meteorological conditions, flow and vortex - to design an ontology which encompasses all models. The application of the ontology, however, seems to be restricted to a niche of urban planning. Data integration using ontologies has been a core use case in semantic web space. 

%Data integration has been used in various fields including healthcare, geography, and social media. An example of such a data integration application is in healthcare where authors \cite{Zhang2018} use ontologies for integration of cancer data to support integrative analysis of the said data. They create a metadata ontology (OCRV) using an upper ontology (BFO), and concepts from NCI Thesaurus (NCIt) and the Time Event Ontology (TEO). The ontology basically models patient data and has attributes like ocrv:deathCause, ocrv:hasDiagnosis, ocrv:hasTumorType, etc. The authors perform SPARQL queries over the modelled data to derive insights about cancer survival. The multiple data sources - patient data, death records, diagnosis - calls for the use of ontologies in modelling.


