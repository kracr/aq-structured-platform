\section{Introduction}
\label{sec:intro}

Air pollution is a significant problem across the major cities of the World and is also a part of the sustainable development goals\footnote{\url{https://www.un.org/sustainabledevelopment/health/}}. Tackling it, however, is a rather complex task as the solution lies at the intersection of society, politics, science and technology. Air pollution has a substantial spatial variance. Even within one city, air quality can vary greatly depending on the locality, proximity to industrial areas, population density, etc. To tackle air pollution, government agencies frame policies for mitigating the sources of pollution, increasing civic engagement and promoting standardized air quality indices such as the Air Quality Index (AQI). In this work, we delve into the latter two aspects, i.e., improving community engagement and studying AQI from the perspective of different social cohorts. This led to a socially relevant AQI, which we named Social AQI (SAQI). 

We conducted an ethnographic survey at six different locations in Delhi, India, which is one of the most polluted cities in the World. At these six locations, we have also deployed air quality sensors to record the hyperlocal AQI. The details of the data collection and the survey results are discussed in Section~\ref{sec:req-analysis}. To model and integrate these disparate and heterogenous datasets, we built a SAQI Ontology (\href{https://saqi-er24.netlify.app/saqi}{saqi}). This ontology~\cite{Ont-NG-2009} captures the relationship between the seemingly unrelated concepts in the datasets, which in turn are used to construct a Knowledge Graph (KG)~\cite{KG-ACMSurvey2021}. A SAQI app is built on top of the KG. This app considers the different social cohorts and their understanding of the AQI. The following are our contributions.
\begin{enumerate}
    \item An ontology (\href{https://saqi-er24.netlify.app/saqi}{saqi}) that models the data from the air quality sensors and the ethnographic survey data. This is built using our Pollution ontology design pattern\footnote{\url{http://ontologydesignpatterns.org/wiki/Submissions:Pollution}}~\cite{Saad-PollutionODP-WOP2021} and is available at \url{https://github.com/anon-DC1E/SAQI-ER-24}. The documentation is at \url{https://https://saqi-er24.netlify.app/saqi}.
    \item A SAQI KG that integrates the air pollution data from local and central sensors with the data from the ethnographic survey. YARRML\footnote{\url{https://rml.io/yarrrml/spec/}} is used to convert the sensor and survey data into a KG. These mappings, along with the SHACL\footnote{\url{https://www.w3.org/TR/shacl/}} constraints and SPARQL queries on the KG, are available at \url{https://doi.org/XX.XXXXX/zenodo.XXXXXXXX}. The SPARQL endpoint over the KG is available at \url{https://https://saqi-er24.netlify.app/saqi/sparql}.
    \item An SAQI App that is built using the KG to encourage community engagement and response. This is available at \url{https://https://saqi-er24.netlify.app/saqi/app}.
    \item The two datasets -- ethnographic survey questionnaire along with the responses and the hyperlocal air quality data from six different locations in Delhi. This is available at \url{https://doi.org/XX.XXXXX/zenodo.XXXXXXXX}. These datasets will be helpful to social scientists, environmental scientists, biophysicists and biochemists.  
\end{enumerate}

%This paper will describe the initial investigation to understand the social aspect of pollution awareness and provide a requirement analysis leading to the development of Air pollution ontology.

%This paper has been formulated as part of the Social AQI (SAQI) project, which observed during initial surveys in Delhi that public engagement with this data-driven governance is not uniform across social cohorts, particularly for the marginalized and neighbourhoods with vulnerable populations. 
%The standardization of the air quality data ignores socio-spatial inequalities, in the form of a standard number for a broad location. In Delhi, each of these sensors cost up to 1-4 crores making it unfeasible to obtain hyperlocal data.

%The surveys targeted six different social groups - teachers, students, mothers of young kids, local shopkeepers, sports academies, and social workers. Exposure to hyperlocal data increases local participation, which can make local institutions respond effectively to their local pollution problems, which the SAQI terms as socializing and localizing the AQI in the neighbourhood.
%Students, teachers and social workers were most aware of AQI and expressed that air pollution should be addressed including at the local level. In contrast, mother of kids had low awareness of the AQI numbers and lacked the ability to obtain and understand air quality data released by the government. At an institutional level, it was observed that even with awareness of the AQI, they avoided taking responsibilities in favour of not adding 'politics' to this issue, e.g. teachers and social workers had a greater understanding of AQI but they tend to attribute responsibilities to different agencies; Teachers assigned responsibility to Municipal Corporation of Delhi (MCD) whereas Social workers and NGO workers preferred agencies of the Union government.

%In an attempt to bridge the gap between institutions and the AQI regime, SAQI purposed various solutions including mid level institutions or volunteers called 'saqis' and to simply increase general literacy levels to raise public awareness. The Air pollution ontology is developed to integrate social data and AQI data from local and government sensors and to also contribute towards building SAQI platform, which is an alternative to AQI monitoring app which caters to different social groups by using multilingual support and penalization for each social group.

%Air pollution ontology consists are divided into three modules, pollution ontology, derived from generic Pollution ODP, trajectory module derived from Trajectory pattern, and an ethnographic module featuring the social make up of localities which are populated using data from surveys. \footnote{http://ontologydesignpatterns.org/wiki/Submissions:Trajectory/Trajectory} \footnote{ontologydesignpatterns.org/wiki/Submissions:Pollution}

%This paper also describes the formulation of a modular ontology to map and integrate pollution data from sensors and survey responses to provide a complete picture of the AQI index and the perception and knowledge of the social community targeted to utilize that index. Ontologies have proven to help integrate data from heterogeneous. In the context of Air Pollution, the SAQI project combines real-time data from sensors and static data entered once and rarely updated, such as pollutants definition, prescribed standards, local perception data from surveys, etc. The following sections describe the requirement analysis for the SAQI ontology, a detailed description of concepts included in the ontology, and a brief overview of the application developed with the help of SAQI ontology.

%1. Ontology - Description of ontology - link
%2. ODP - 1-2 line desc. - link (ODP repo) - (ontologydesignpatterns.org/wiki/Submissions:Pollution)
%3. App - saqi app (host it to ***** website)
%4. Survey+ Hyperlocal data - Link to access (not on Gdrive) - zenodo (open liscence) surrvey responses csv and form, 6 sensor location.
%Useful for - researchers (SS people), delhi pollution, or pollution awareness across different social groups in Delhi.




