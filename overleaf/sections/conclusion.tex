\section{Conclusion and Future Work}
\label{sec:conclusion}

We proposed Social Air Quality Index (SAQI), a new approach to increase awareness about air pollution,  democratize AQI knowledge at the neighbourhood level and improve community participation. We developed a SAQI ontology to model the local and central air quality data and the ethnographic survey data from six localities in Delhi. The sensor and survey data are integrated into a KG using YARRRML mappings. We designed a neighbourhood pollution dashboard (SAQI app) that uses the KG to answer questions related to air pollution and the steps that each social cohort (students, teachers, mothers of young kids, athletes, etc.) can take to mitigate the effects of air pollution. The app has received positive feedback on its usability and utility. All these resources are publicly available under an Apache License 2.0 at \url{https://github.com/anon-DC1E/SAQI-ER-24} and \url{https://doi.org/XX.XXXXX/zenodo.XXXXXXXX}. 

In the future, we plan to incorporate more social characteristics from the neighbourhood into the ontology and the SAQI app, such as addresses and contacts for local government officials and the local pollution generating sources. We also plan to introduce graphical visualization in the app to explain the distribution and variation of AQI with respect to the time of the day, the seasonal variations and its correlation to activities in the neighbourhood.

%\textbf{Acknowledgement.} The ***** ***** for ******** ************ (CAI), *****-*****, India has partially supported this work.